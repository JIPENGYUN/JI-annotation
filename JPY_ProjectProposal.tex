\documentclass{report}

\input Latex_macros/Definitionen.tex

\usepackage{a4}
\usepackage[all]{xy}
\usepackage{enumerate}
\usepackage{xr}


\begin{document}
	
	\title{Initial Document\\Heuristics for SAT solvers}
	\author{Ji Pengyun\\
		\href{http://www.swan.ac.uk/compsci/}{Computer Science Department}, \href{http://www.swan.ac.uk/science/}{College of Science}\\
		\href{http://www.swan.ac.uk/}{Swansea University, UK}\\
		%{\small \url{http://www.swan.ac.uk/NOT_YET}}
	}
	
	\maketitle
	
	\tableofcontents
	
	\chapter{General plans}
	\label{cha:Generalpreparations}
	
	
	To accomplish the initial report, planning uses ``Todos'':
	\begin{enumerate}
		\item To meet Oliver twice a week to communicate:
		\begin{enumerate}[Week 1]
			\item 21/3/2016 - 27/3/2016: XXX
		\end{enumerate}
		\item Integrate Latex-structure, either by appropriate copying or by using directly Annotations.
		\item Looking into the paper ``Fundaments of Branching Heuristics'' of Handbook Chapter 7 (O.Kullmann, 2009). Start with
		\begin{enumerate}
			\item Missing proofs of Theorems:
			\begin{enumerate}
				\item XXX
			\end{enumerate}
		\end{enumerate}
		\item Working through Chapter 7:
		\begin{enumerate}
			\item {\bf The Definitions of handbook chapter.}
			
			{\defi $\mathcal{BT} :=\bigcup_{k\in \mathbb{N}}(\mathbb{R}_{>0})^k$}
			denotes the set of {\bf branching tuples}.
			
			Remarks:
			
			\begin{enumerate}
				\item  Basic measurements for branching tuples are minimum $\min: \mathcal{BT} \to \mathbb{R}_{>0}$,
				maximum $\max : \mathcal{BT} \to \mathbb{R}_{>0}$, sum $\Sigma : \mathcal{BT} \to \mathbb{R}_{>0}$, and width $|| : \mathcal{BT} \to \mathbb{N}$.
				The minimum $\min(a)$ of a branching tuple is a ``worst-case view'', while
				max($a$) is a ``best-case view''. In general, disregarding the values, the
				larger $|a|$, i.e., the wider the branching is, the worse it is.
				
				\item The set of branching tuples of width $k$ is ${\mathcal{BT}}^{{(k)}}:= \{t \in \mathcal{BT} : |t| = k\}$,
				which is a {\it cone}, that is for $a \in \mathcal{BT}^{(k)}$ and $\lambda \in \mathbb{R}_{>0}$ we have $\lambda \cdot t \in \mathcal{BT}^{(k)} $, and for $a,b \in \mathcal{BT}^{(k)}$ we have $a + b \in \mathcal{BT}^{(k)}$.
				
				\item Branching tuples of width 1, which do not represent ``real branchings'' but
				``reductions'', are convenient to allow.
				
				\item One could also allow the empty branching tuple as well as the zero branching tuple (0) (of width 1), but for the sake of simplicity we abstain from
				such systematic extensions here.
			\end{enumerate}
			
			
			{\defi Define $\chi _k : \mathcal{BT} \times \mathbb{R}_{>0} \to \mathbb{R}_{>0}$ by $\chi (t,x) :=
				\sum^{|t|}_{i=1} x^{-t_i}$. Observe
				that for each $t \in \mathcal{BT}$ the map $\chi (t,-) : \mathbb{R}_{>0} \to \mathbb{R}_{>0}$ is strictly decreasing with
				$\chi (t,1) = |t| \geq 1$ and $\lim _{x\to \infty} \chi (t,x) = 0$. Now $\tau : \mathcal{BT} \to \mathbb{R}_{\geq 1} $ is defined as the
				unique $\tau (t) := x_0 \in\mathbb{R}_{\geq 1} $ such that $\chi (t)(x_0 ) = 1$ holds.}
			
			By definition we have $\tau (t) \geq 1$, with $\tau (t) = 1 \Leftrightarrow |t| = 1$. For $k \in \mathbb{N} $ we denote
			by $\tau_k : \mathcal{BT}^{(k)}
			\to \mathbb{R}_{\geq 1} $ the $\tau $-function restricted to branching tuples of width $k$.
			
			
			{\lem For every $a \in \mathcal{BT} , k \in \mathbb{N}$ and $\lambda \in \mathbb{R}_{>0}$ we have:
				
				\begin{enumerate}
					\item[1.]$\tau (\lambda \cdot a) = \tau (a)^{1/\lambda} $.
					
					\item[2.] $\tau_k (\vec{1}) = k$.
					
					\item[3.] $\tau_k$ for $k \geq 2$ is strictly decreasing in each component.
					
					\item[4.] $\tau_k$ is symmetric, that is, invariant under permutation of branching tuples.
					
					\item[5.] $\tau (a)^{\min(a)} \leq |a| \leq \tau (a)^{\max(a)}$, that is, $|a|^{1/\max(a)} \leq \tau (a) \leq |a|^{1/\min(a)}$.
					
					\item[6.] $\lim_{\lambda \to 0} \tau (a;(\lambda )) = \infty$ and $\lim _{\lambda \to \infty }\tau (a;(\lambda )) = \tau (a)$.
				\end{enumerate}}
				
				The $\tau $-function fulfils powerful convexity properties, from which non-trivial
				further properties will follow. A function $f : C \to \mathbb{R}$ defined on some convex
				subset $C \subseteq \mathbb{R}^k$ is called ``strictly convex'' if for all $x,y \in C$ and $0 < \lambda < 1 $ holds
				$f(\lambda x+(1-\lambda )y) < \lambda f(x)+(1-\lambda )f(y)$; furthermore $f$ is called ``strictly concave''
				if $-f$ is strictly convex. By definition $\tau_1$ is just the constant function with value
				1, and so doesn't need to be considered here.
				
				{\bf Bounds on the tau-function}
				
				Just from being symmetric and strictly convex it follows, that $\tau_k (a)$ for tuples a
				with a given fixed sum $\Sigma (a) = s$ attains its strict minimum for the constant tuple
				(with entries $\frac sk $); Thus, using $\mathfrak{A}(t) :=
				\sum(t)/|t|$ for the arithmetic mean of a branching tuple, we have $\tau (\mathfrak{A}(t) \cdot \vec{1}) \leq \tau (t)$ (with
				strict inequality iff $t$ is not constant).
				
				\begin{enumerate}
					\item The {\it arithmetic mean}, the {\it geometric mean}, and the {\it harmonic mean} of
					branching tuples $t$ are denoted respectively by $\mathfrak{A}(t) :=\frac1{|t|} \Sigma (t) =
					\frac1{|t|}\sum^{|t|}_{i=1} t_i$,
					$\mathfrak{G}(t) :=\sqrt[|t|]{\prod^n_{i=1} t_ i}$
					and $\mathfrak{H}(t) := |t|/(\sum^n_{i=1}\frac1{t_i})$. We recall the well-known
					fundamental inequality between these means: $\mathfrak{H}(t) \leq \mathfrak{G}(t) \leq \mathfrak{A}(t)$ (where
					strict inequalities hold iff $t$ is not constant).
					
					\item More generally we have the {\it power means} for $\alpha \in \overline{\mathbb{R}}$ given by $\mathfrak{M} _\alpha (t) :=(\frac1{|t|}\sum^{|t|}_{i=1} t^\alpha_i)^{1/\alpha}$
					for $\alpha \notin \{-\infty ,0,+\infty \}$, while we set $\mathfrak{M} _{-\infty} (t) := \min(t)$,
					$\mathfrak{M}_ 0 (t) := \mathfrak{G}(t)$ and $\mathfrak{M} _{+\infty} (t) := \max(t)$. By definition we have $\mathfrak{M}_{{-1}} (t) =
					\mathfrak{H}(t)$ and $\mathfrak{M}_ 1 (t) = \mathfrak{A}(t)$. In generalisation of the above fundamental inequality we have for $\alpha $, $\alpha ' \in \overline{\mathbb{R}}$ with $\alpha < \alpha '$ the inequality $\mathfrak{M} _\alpha (t) \leq \mathfrak{M} _{\alpha ' } (t)$,
					which is strict iff $t$ is not constant.
				\end{enumerate}
				{\defi Consider $k \in \mathbb{N}$. A {\bf mean} is a map $M : \mathcal{BT}^{(k)}
					\to\mathbb{R}_{ >0}$
					which is continuous, strictly monotonic increasing in each coordinate, symmetric
					(i.e., invariant under permutation of the arguments) and ``consistent'', that is,
					$\min(a) \leq M(a) \leq \max(a)$ for $a \in \mathcal{BT}^{(k)}$. A mean $M$ is {\bf homogeneous} if $M $ is
					positive homogeneous, i.e., for $\lambda \in \mathbb{R}_{ >0}$ and $a \in \mathcal{BT}^{(k)}$
					we have $M(\lambda \cdot a) = \lambda \cdot M(a)$.}
				
				All power means are homogeneous means. Yet $k$-ary means $M$ are only
				defined for tuples of positive real numbers $a \in \mathbb{R}_{ >0}^ k
				$, and we extend this as follows
				to allow arguments 0 or $+\infty $, using the extended real line $\overline{\mathbb{R}} = {\mathbb{R}} \cup \{\pm \infty \}$. We
				say that $M$ is defined for $a \in \overline{\mathbb{R}}_{ \leq0}^ k$
				(allowing positions to be 0 or $+\infty $) if the
				limit $\lim _{a ' \to a,a' \in \mathbb{R}_{>0}^k} M(a')$ exists in $\overline{\mathbb{R}}$, and we denote this limit by $M(a)$ (if the
				limit exists, then it is unique). Power means $ \mathfrak{M} _\lambda (a)$ for $\lambda \neq0$ are defined for all
				$a \in \overline{\mathbb{R}}_{ \leq0}^ k$, while $\mathfrak{M}_ 0 (a) = \mathfrak{G}(a)$ is defined iff there are no indices $i,j $ with $a_i= 0$
				and $a _j = \infty $.
				
				{\defi Consider a mean $M : \mathcal{BT}^{(k)}
					\to {\mathbb{R}}_{>0}$. We say that $M$ is $\infty ${\bf -dominated} resp. 0{\bf -dominated} if for every $a \in \overline{\mathbb{R}}_{ \geq0}^ k$, such that an index $i$ with
					$a _i = \infty$ resp. $a_ i = 0$ exists, $M(a)$ is defined with $M(a) = \infty$ resp. $M(a) = 0$. On
					the other hand, $M$ {\bf ignores} $\infty$ resp. {\bf ignores} 0 if $M(a;(\infty ))$ resp. $M(a;(0))$ is
					defined iff $M(a)$ is defined with $M(a;(\infty )) = M(a)$ resp. $M(a;(0)) = M(a)$.}
				
				
				Power means $\mathfrak{M} _\lambda$ with $\lambda > 0$ are $\infty $-dominated and ignore 0, while for $\lambda < 0$
				they are 0-dominated and ignore $\infty $. The borderline case $\mathfrak{M}_ 0 = \mathfrak{G}$ is $\infty $-dominated
				as well as 0-dominated if only tuples are considered for which $\mathfrak{G}$ is defined (and
				thus we do not have to evaluate ``$0 \cdot \infty $'').
				
				Another important properties of means is convexity resp. concavity. Power
				means $\mathfrak{M }_\alpha$ with $\alpha > 1$ are strictly convex, while power means $\mathfrak{M}_\alpha$ with $\alpha < 1$ are
				strictly concave; the borderline case $\mathfrak{M}_1 = \mathfrak{A}$ is linear (convex and concave).
				
				{\lem For every concave mean $M$ we have $\mathfrak{M} \leq \mathfrak{A}$.}
				
				{\prf By Jensen's inequality we have $M(a) =\sum_{\pi \in S_k}\frac1{k!} M(\pi * a) \leq
					M( \sum_{\pi \in S_k}\frac1{k!}\cdot (\pi * a)) = M(\mathfrak{A}(a) \cdot \vec{1}) = \mathfrak{A}(a)$.} \hfill  $\Box$
				
				{\bf Jensen's Inequality}
				
				If $p_1,\ldots,p_n$ are positive numbers which sum to 1 and $f$ is a real
				continuous function that is convex, then
				\begin{equation}
				f\left(\sum^n_{i=1}p_ix_i\right) \leq \sum^{n}_{i=1}p_if(x_i).
				\end{equation}
				
				If $f$ is concave, then the inequality reverses, giving
				\begin{equation}
				f\left(\sum^n_{i=1}p_ix_i\right) \geq \sum^{n}_{i=1}p_if(x_i).
				\end{equation}
				
				The special case of equal $p_i=1/n$ with the concave function $\ln x$ gives
				\begin{equation}
				\ln\left(\frac1n\sum^n_{i=1}x_i\right) \geq \frac1n\sum^{n}_{i=1}\ln x_i,
				\end{equation}
				
				
				which can be exponentiated to give the arithmetic mean-geometric mean inequality
				\begin{equation}
				\frac{x_1+x_2+\ldots + x_n}{n}\geq \sqrt[n]{x_1x_2\cdots x_n}.
				\end{equation}
				
				Here, equality holds iff $x_1=x_2=\ldots = x_n$.
				
				
				
				
			\end{enumerate}
			\item Algorithm and efficient implementation of $\tau(a,b)$ XXX
		\end{enumerate}
		
		
		
	\end{document}