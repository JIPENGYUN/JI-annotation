\documentclass{report}

\input Latex_macros/Definitionen.tex

\usepackage{a4}
\usepackage[all]{xy}
\usepackage{enumerate}
\usepackage{xr}


\begin{document}

\title{Initial Document\\Heuristics for SAT solvers}
\author{Ji Pengyun\\
  \href{http://www.swan.ac.uk/compsci/}{Computer Science Department}, \href{http://www.swan.ac.uk/science/}{College of Science}\\
   \href{http://www.swan.ac.uk/}{Swansea University, UK}\\
  %{\small \url{http://www.swan.ac.uk/NOT_YET}}
}

\maketitle

\tableofcontents

\chapter{General plans}
\label{cha:Generalpreparations}


To accomplish the initial report, planning uses ``Todos'':
\begin{enumerate}
	\item To meet Oliver twice a week to communicate:
	\begin{enumerate}[Week 1]
		\item 21/3/2016 - 27/3/2016: XXX
	\end{enumerate}
	\item Integrate Latex-structure, either by appropriate copying or by using directly Annotations.
	\item Looking into the paper ``Fundaments of Branching Heuristics'' of Handbook Chapter 7 (O.Kullmann, 2009). Start with
	\begin{enumerate}
		\item Missing proofs of Theorems:
		\begin{enumerate}
			\item XXX
		\end{enumerate}
	\end{enumerate}
	\item Working through Chapter 7:
	\begin{enumerate}
		\item XXX
	\end{enumerate}
	\item Algorithm and efficient implementation of $\tau(a,b)$ XXX
\end{enumerate}



\end{document}