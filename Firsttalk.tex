\documentclass{article}
\usepackage{latexsym}
\usepackage{graphicx}
\usepackage{amsmath}
\usepackage{amssymb}
\usepackage{amsthm}
\usepackage{bm}
\usepackage{float}
\usepackage{array}
\usepackage{multirow}

\newcommand{\yi}{\raisebox{-1pt}{\includegraphics{q1.eps}}}
\newcommand{\er}{\raisebox{-1pt}{\includegraphics{q2.eps}}}
\newcommand{\san}{\raisebox{-1pt}{\includegraphics{q3.eps}}}
\newcommand{\si}{\raisebox{-1pt}{\includegraphics{q4.eps}}}
\newtheorem*{mydef}{Definition}
\newtheorem*{thm}{Theorem}

\begin{document}

\section{First talk at MRes seminar}
\label{sec:FirsttalkMRes}

\begin{mydef}\label{def:cartesianprod}
  Suppose $A$ and $B$ are sets. Then the \textbf{product} of $A$ and $B$, denoted by $A \times B$, is the set of all ordered pairs in which the first component is an element of $A$ and the second is an element of $B$. In other words,
  \begin{displaymath}
    A \times B := \{(a,b)  |  a \in A \text{ and }  b \in B \}.
  \end{displaymath}
\end{mydef}

\begin{thm}\label{thm:propprod}
  Consider sets $A, B, C, D$.
  \begin{enumerate}
  \item\label{thm:propprod5} $A \times \emptyset = \emptyset \times A = \emptyset$.
  \item\label{thm:propprod1} $A \times (B \cap C) = (A\times B) \cap (A \times C)$.
  \item\label{thm:propprod2} $A \times (B \cup C) = (A \times B) \cup (A \times C)$.
  \item\label{thm:propprod3} $(A \times B) \cap (C \times D) = (A \cap C) \times (B \cap D)$.
  \item\label{thm:propprod4} $(A \times B) \cup (C \times D) \subseteq (A \cup C) \times (B \cup D)$.
  \end{enumerate}
\end{thm}
\begin{proof}
\begin{figure}[H]
  \centering
  \includegraphics[width=8.5cm]{fig1}\\
  
  \bigskip
  
  \includegraphics[width=8.5cm]{fig2}\\
\end{figure}

Part \ref{thm:propprod4}:  $(A \times B) \cup (C \times D) \subseteq (A \cup C) \times (B \cup D)$.\\
First introduce arbitrary objects $x$, $y$ that turn out to be an ordered pair, say $(x, y)$.\\
Let $(x,y)$ be an arbitrary element of $(A \times B) \cup (C \times D)$.\\
Then either $(x,y) \in (A \times B)$ or $(x,y) \in (C \times D)$.\\
\newline
Case 1. $(x,y) \in (A \times B)$. Then by definition $x \in A$ and $y \in B$.\\
Apparently $x \in (A \cup C)$ and $y \in (B \cup D)$.\\
Therefore, $(x,y) \in (A \cup C) \times (B \cup D)$.\\
\newline
Case 2. $(x,y) \in (C \times D)$. That's the similar argument shows that $x \in (A \cup C)$ and $y \in (B \cup D)$.\\
Therefore, $(x,y) \in (A \cup C) \times (B \cup D)$.\\
\newline
As $(x,y)$ was an arbitrary element of $(A \times B) \cup (C \times D)$, this follows that $(A \times B) \cup (C \times D) \subseteq (A \cup C) \times (B \cup D)$.

$\hfill\blacksquare$
\end{proof}
Remarks:\\
For Part \ref{thm:propprod4} the reverse direction does not hold $(A \cup C) \times (B \cup D) \subseteq (A \times B) \cup (C \times D)$.\\
\begin{proof}
Consider $(x,y)$ be an arbitrary element of $(A \cup C) \times (B \cup D)$. Then $(x,y) \in (A \cup C) \times (B \cup D)$.\\
By definition\ref{def:cartesianprod} that $x \in (A \cup C)$ and $y \in (B \cup D)$,\\
either $x \in A$ or $C$ and either $y \in B$ or $D$.\\

Case 1. $x \in A$ and $y \in B$, then $(x,y) \in (A \times B)$.\\
So $(x,y) \in (A \times B) \cup (C \times D)$.\\
Since $(x,y) \in (A \cup C) \times (B \cup D)$.\\
Therefore, $(A \cup C) \times (B \cup D) \subseteq (A \times B) \cup (C \times D)$ hold in first case.\\

Case 2. $x \in C$ and $y \in D$, then $(x,y) \in (C \times D)$.\\
Here follow by basically similar arguments with case 1.\\
Therefore, $(A \cup C) \times (B \cup D) \subseteq (A \times B) \cup (C \times D)$ hold in second case.\\

Case 3. $x \in A$ and $y \in D$, then $(x,y) \in (A \times D)$.\\
However $A \times D$ is not a subset of $(A \times B) \cup (C \times D)$,\\
Since $(x,y) \in (A \cup C) \times (B \cup D)$,\\
Therefore $(A \cup C) \times (B \cup D) \varsubsetneq (A \times B) \cup (C \times D)$.\\
\newline
A contradiction proof is given.

$\hfill\blacksquare$
\end{proof}



\end{document}
